\begin{frame}[containsverbatim]
  \frametitle{\secname}
  \framesubtitle{\subsecname~: L'opérateur $<<$} 

  Cet opérateur retourne le décalage binaire à gauche de son opérande gauche de son opérande droit nombre de fois. 
  \par
  Cet opérateur ne modifie pas la valeur de ses opérandes.
  \par
  Un décalage binaire à gauche signifie décaler tous les bits d'un rang vers la gauche, dans une représentation binaire de la variable. Le
  bit le plus à gauche est perdu, et le bit le plus à droite est à 0.
  \vspace{0.3cm}
  \par
  Par exemple, pour un \verb|unsigned char|, contenant 149, soit en binaire \texttt{10010101} (128 + 16 + 4 + 1), si on effectue 1 décalage
  à gauche, le nombre obtenu est 42, soit en binaire \texttt{00101010}.
  \vspace{0.3cm}
  \par
  \begin{exampleblock}{Example d'utilisation de l'opérateur $<<$}
    \begin{verbatim}
int a = 149, b;
b = a << 2; // la valeur de a subit deux décalages à gauche
            // b reçoit 84 (64 + 16 + 4)\end{verbatim}
  \end{exampleblock}
\end{frame}

