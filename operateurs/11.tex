\begin{frame}[containsverbatim]
  \frametitle{\secname}
  \framesubtitle{\subsecname~: L'opérateur = (affectation)~I} 

  Nous l'avons déjà abordé précédemment, mais cet opérateur est plus complexe qu'à premier abord.
  \vspace{0.3cm}
  \par
  Cet opérateur effectue l'affectation de la valeur retournée par son opérande gauche à son opérande droit (parfois appelé \textit{LVALUE}), \textbf{puis retourne la valeur affectée}.
  \vspace{0.3cm}
  \par
  Cela signifie que l'action de cet opérateur ne se résume pas à l'affectation de valeur, et génère elle-même une valeur, égale à la valeur affectée.
  \vspace{0.3cm}
  \par
  Autrement dit, la ligne de code suivante génère la valeur 3~:
  \begin{exampleblock}{L'affectation retourne une valeur}
    \begin{verbatim}
char a;
a = 3; // cette expression retourne la valeur 3
       // en plus d'affecter 3 à la variable a\end{verbatim}
  \end{exampleblock}
\end{frame}

