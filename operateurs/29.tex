\begin{frame}[containsverbatim]
  \frametitle{\secname}
  \framesubtitle{\subsecname~: L'opérateur ||} 

  Cet opérateur représente l'opérateur ``OU logique''. Il retourne 1 si au moins l'un de ses deux opérandes vaut vrai (valeur non nulle), 
  sinon 0 (faux).
  \par
  Cet opérateur n'altère pas ses opérandes.
  \vspace{0.5cm}
    \begin{center}
    \begin{tabular}{|c|c|c|}
      \hline
      || & 0 & 1 \\
      \hline
      0  & 0 & 1 \\
      \hline
      1  & 1 & 1 \\
      \hline
    \end{tabular}
  \end{center}
  \begin{exampleblock}{Exemple d'utilisation de l'opérateur ||}
    \begin{verbatim}
#include <stdbool.h>
int a = 5, b = 8;
bool c = (a < 7) || (b > 10); // c vaut 1 (vrai)
    \end{verbatim}
  \end{exampleblock}
\end{frame}

