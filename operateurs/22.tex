\begin{frame}[containsverbatim]
  \frametitle{\secname}
  \framesubtitle{\subsecname~: L'opérateur ==} 

  Cet opérateur retourne 1 (vrai) quand les deux opérandes sont égaux, 0 (faux) sinon. Cet opérande n'altère pas ses opérandes.
  \par
  \begin{alertblock}{L'erreur classique !}
    Une erreur classique des débutants est de ne mettre qu'un seul symbole = pour effectuer le test d'égalité. Dans ce cas, c'est l'affectation qui
    est faite, et qui renvoie, comme vu précédemment, l'opérande droit.
  \end{alertblock}
  \begin{exampleblock}{Utilisation de l'opérateur ==}
    \begin{verbatim}
#include <stdbool.h>
int a = 5, b = 5;
char c = '5';
bool test;   
test = a == b; // test vaut 1 (vrai)
test = a == c; // test vaut 0 (faux) car '5' = 53  \end{verbatim}
  \end{exampleblock}
\end{frame}

