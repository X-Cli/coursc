\begin{frame}[containsverbatim]
  \frametitle{\secname}
  \framesubtitle{\subsecname~: Les opérateurs de calculs et d'affectations simultanés} 

  La plupart des opérateurs de calcul possèdent un raccourci quand le résultat de l'opération est affecté à une variable qui se trouve être son opérande gauche.
  \vspace{0.3cm}
  \par
  Dans ce cas, on utilise également un opérateur d'arité 2, dont la syntaxe est le symbole de l'opération désirée suivi du symbole ``=''. L'opérande de gauche subit 
  l'opération avec l'opérande droit et reçoit le résultat de cette opération.
  \begin{exampleblock}{Exemple de l'addition}
    \begin{verbatim}
int a = 5;
a += 2; // équivalent à a = a + 2\end{verbatim}
  \end{exampleblock}
  Voici la liste des opérateurs permettant ce raccourci~:
  \begin{center}
    \begin{tabular}{|>{\centering\arraybackslash}m{1.5cm}|>{\centering\arraybackslash}m{1.5cm}|>{\centering\arraybackslash}m{1.5cm}|>{\centering\arraybackslash}m{1.5cm}|>{\centering\arraybackslash}m{1.5cm}|}
      \hline
      += & -= & *= & /= & \%= \\
      \hline
      \&= & |= & \^{ }= & $<<$= & $>>$= \\
      \hline
    \end{tabular}
  \end{center}  
\end{frame}

