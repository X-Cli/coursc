\begin{frame}[containsverbatim]
  \frametitle{\secname}
  \framesubtitle{\subsecname~: L'opérateur ?:} 

  Cet opérateur utilise 3 opérandes. Il renvoie le deuxième opérande si la valeur du premier est vrai (valeur non nulle), et sinon il renvoie le troisième. 
  \par 
  L'opérande qui n'est pas renvoyé n'est pas évalué (et son code n'est donc pas exécuté).
  \vspace{0.3cm}
  \begin{exampleblock}{Exemple d'utilisation de l'opérateur ternaire ?:}
    \begin{verbatim}
int a = 2, b;
b = a < 5 ? 3 : 6; // b reçoit 3
b = (a < 1 || a > 5) ? 5: 9; // b reçoit 9
b = a ? a + 5 : 0; // si a non nul, b reçoit a + 5, sinon 0\end{verbatim}
  \end{exampleblock}
\end{frame}

