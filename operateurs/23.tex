\begin{frame}[containsverbatim]
  \frametitle{\secname}
  \framesubtitle{\subsecname~: L'opérateur !=} 

  Cet opérateur retourne 0 (faux) quand les deux opérandes sont égaux, 1 (vrai) sinon. Cet opérande n'altère pas ses opérandes.
  \vspace{0.3cm}
  L'opérateur d'inégalité est équivalent à l'utilisation de l'opérateur d'égalité dont le résultat serait inversé à l'aide de l'opérateur
  non logique~: !
  \vspace{0.3cm}
  \begin{exampleblock}{Exemple d'utilisation de l'opérateur !=}
    \begin{verbatim}
#include <stdbool.h>
int a = 5, b = 7;
bool c;
c = a != b; // c vaut 1 (vrai)
c = !(a == b); // c vaut 1 (vrai)
c = !(a != b); // c vaut 0 (faux)\end{verbatim}
  \end{exampleblock}
\end{frame}

