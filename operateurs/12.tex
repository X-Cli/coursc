\begin{frame}[containsverbatim]
  \frametitle{\secname}
  \framesubtitle{\subsecname~: L'opérateur = (affectation)~II} 

  Du fait que l'affectation retourne une valeur, il est possible de chainer les affectations ou d'effectuer des affectations au milieu d'autres
  opérations.
  \par
  \vspace{0.3cm}
  {\small\begin{exampleblock}{Chainage d'affectations}
    \begin{verbatim}
int a, b;
a = b = 3;\end{verbatim}
    \verb|a| reçoit le résultat de son opérande droit \verb|b = 3|, soit la valeur 3. 
    \par
    \verb|b| reçoit 3.
  \end{exampleblock}
  \begin{exampleblock}{Affectation au milieu d'une opération}
    \begin{verbatim}
int a, b;
b = (a = 4) + 2; //a vaut 4, et b vaut 6\end{verbatim}    
    \verb|a| reçoit 4, et l'affectation renvoie 4 qui est sommé à 2, formant 6 qui est affecté à \verb|b|.
  \end{exampleblock}}
\end{frame}

