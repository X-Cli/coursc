\begin{frame}[containsverbatim]
  \frametitle{\secname}
  \framesubtitle{\subsecname~: Un point sur les parenthèses} 

  En C, lorsque plusieurs opérateurs sont chainés dans une même expression, il existe des règles de priorité de résolution, tout comme en mathématiques.
  \vspace{0.5cm}
  \par
  Nous verrons plus tard l'ordre de priorité des opérateurs, qui est parfois contre-intuitif. Les parenthèses peuvent être utilisées pour retirer
  des ambiguïtés (simplifier une opération en la rendant plus lisible) ou pour forcer un changement de priorité\ldots comme en mathématiques.
  \vspace{0.5cm}
  \begin{exampleblock}{Exemple d'utilisation des parenthèses}
    \par
    La multiplication est prioritaire sur l'addition, en C. L'opération $5 * 3 + 2$ vaut 17 en mathématiques, comme en C.
    \par
    On peut modifier l'ordre à l'aide de parenthèse de manière classique~:
    \begin{verbatim}
int a = 5 * (3 + 2); // a vaut 25 \end{verbatim}   
  \end{exampleblock}
\end{frame}

