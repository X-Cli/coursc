\begin{frame}[containsverbatim]
  \frametitle{\secname}
  \framesubtitle{\subsecname~: L'opérateur -\relax-} 
  
  Cet opérateur peut \textit{préfixer} ou \textit{suffixer} son opérande. 
  \vspace{0.3cm}
  \par
  Son effet est la \textbf{décrémentation}~: il diminue la valeur de son opérande de 1.
  \vspace{0.3cm}
  \par
  Si l'opérateur préfixe l'opérande, l'opérande est décrémenté, puis sa valeur est retournée, afin que l'expression qui les contient soit évaluée à son tour.
  \vspace{0.3cm}
  \par
  Si l'opérateur suffixe l'opérande, la valeur est retournée, puis l'opérande voit sa valeur décrémentée.
  \vspace{0.3cm}
  \begin{exampleblock}{Exemples d'utilisation de -\relax-}
    \begin{verbatim}
int a = 3;
int b = a-- + 4; // b reçoit 7, et a vaut maintenant 2
int c = --a + 4; // a vaut 1, et c vaut 5\end{verbatim}        
  \end{exampleblock}
\end{frame}

