\begin{frame}[containsverbatim]
  \frametitle{\secname}
  \framesubtitle{\subsecname~: L'opérateur |} 

  Cet opérateur effectue un ``OU binaire'' (OR) de son opérateur gauche avec son opérateur droit. Cet opérateur n'altère pas ses opérandes.
  \vspace{0.3cm}
  \par
  Le OR binaire s'effectue bit à bit entre les deux opérandes.
  \par
  \begin{center}
    \begin{tabular}{|c|c|c|}
      \hline
      | & 0 & 1 \\
      \hline
      0  & 0 & 1 \\
      \hline
      1  & 1 & 1 \\
      \hline
    \end{tabular}
  \end{center}
  \begin{exampleblock}{Exemple d'utilisation de l'opérateur |}
    \begin{verbatim}
int a = 42, b = 12, c;
c = a | b; // c reçoit 00101010 | 00001100, 
           // soit 00101110 => 46\end{verbatim}
  \end{exampleblock}
\end{frame}

