\begin{frame}[containsverbatim]
  \frametitle{\secname}
  \framesubtitle{\subsecname~: L'opérateur ++} 
  
  Cet opérateur peut \textit{préfixer} ou \textit{suffixer} son opérande. 
  \vspace{0.3cm}
  \par
  Son effet est l'\textbf{incrémentation}~: il augmente la valeur de son opérande de 1.
  \vspace{0.3cm}
  \par
  Si l'opérateur préfixe l'opérande, l'opérande est incrémenté, puis sa valeur est retournée, afin que l'expression qui les contient soit évaluée à son tour.
  \vspace{0.3cm}
  \par
  Si l'opérateur suffixe l'opérande, la valeur est retournée, puis l'opérande voit sa valeur incrémentée.
  \vspace{0.3cm}
  \begin{exampleblock}{Exemples d'utilisation de ++}
    \begin{verbatim}
int a = 3;
int b = a++ + 4; // b reçoit 7, et a vaut maintenant 4
int c = ++a + 4; // a vaut 5, et c vaut 9\end{verbatim}        
  \end{exampleblock}
\end{frame}

