\begin{frame}[containsverbatim]
  \frametitle{\secname}
  \framesubtitle{\subsecname~: Les opérateur \&\& et ||~: Débranchement~I} 

  {\small\begin{alertblock}{Débranchement}
    Les opérateurs logiques ET et OU sont débranchants.
    \par
    Cela signifie que pour des raisons de performances, il a été décidé que si le premier opérande permet de décider à lui seul du résultat d'une
    opération impliquant l'un de ces deux opérateurs, alors le deuxième opérande n'est pas évalué !
    \vspace{0.3cm}
    \par
    Ne pas être évalué signifie que l'expression n'est pas du tout exécutée, et si des opérations altérant des variables ou des appels de fonction
    sont effectués dans le deuxième opérande, ces instructions n'ont pas lieu.
    \vspace{0.3cm}
    \par
    Le cas se produit si le premier opérande d'un ET logique est faux~: quelque soit la valeur du second résultat, le résultat est nécessairement
    faux.
    \par
    Le cas se produit également si le premier opérande d'un OU logique est faux~: quelque soit la valeur du second résultat, le résultat est
    nécessairement vrai.
  \end{alertblock}}
\end{frame}

