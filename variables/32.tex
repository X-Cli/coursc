\begin{frame}[containsverbatim]
  \frametitle{\secname}
  \framesubtitle{\subsecname~I} 

  Cette multitude de types, primitifs, aliasés ainsi que les types composés ne sont pas nécessairement compatibles entre eux. Selon le vieil adage, 
  on ne peut comparer des pommes et des poires\ldots
  \vspace{0.3cm}
  \par 
  Nous reviendrons plus tard sur l'opérateur d'affectation ; considérons pour l'instant que le symbole \verb|=| permet de stocker la valeur ou
  expression à droite du symbole, dans la ``variable'' à gauche de ce dernier (on appelle cette partie une \verb|LVALUE|).
  \vspace{0.3cm}
  \begin{exampleblock}{Affectation d'une valeur à une variable}
    \begin{verbatim}
int a, b;
a = 5; // a vaut 5, b est indéfini
b = a; // a vaut toujours 5, mais b aussi
a = 7; // a vaut désormais 7, b reste à la valeur 5
       //                     affectée précédemment \end{verbatim}
  \end{exampleblock}  
\end{frame}
