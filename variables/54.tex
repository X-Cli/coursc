\begin{frame}[containsverbatim]
  \frametitle{\secname}
  \framesubtitle{\subsecname~: Les énumérations~VI} 

  Il est possible de forcer la numérotation des constantes d'une énumération. Cela peut être utile, notamment lorsqu'on
  définit un \textit{champ de bits}. Nous reviendrons sur les champs de bits lorsque nous aborderons les opérateurs
  binaires.
  \vspace{0.3cm}
  \par
  Notons simplement la syntaxe pour le moment.
  \begin{exampleblock}{Champ de bits avec des énumérations}
    \begin{verbatim}
enum OptionsVoiture {
  opt_ABS = 1,
  opt_Climatisation = 1<<1,
  opt_VitresElectriques = 1<<2,
  opt_FermetureCentralisee = 1<<3
};
enum OptionsVoiture myOptions;
myOptions = opt_ABS | opt_VitresEletriques;\end{verbatim}
  \end{exampleblock}
\end{frame}

