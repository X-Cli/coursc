\begin{frame}
  \frametitle{\secname}
  \framesubtitle{\subsecname~: Les structures~I} 

  Il est possible d'agglomérer plusieurs informations au sein d'un même type de données.
  \vspace{0.3cm}
  \par
  Pour cela, on définit un nouveau type de données, une \textbf{structure}, en précisant quelles informations
  la compose.
  \vspace{0.3cm}
  \par
  Une structure peut regrouper tous types de données précédemment déclarés, que ce soit des types primitifs, 
  d'autres structures, des énumérations, des unions, des pointeurs de tous types, des tableaux \ldots
  \vspace{0.3cm}
  \par
  La structure agit comme un regroupement de taille arbitraire de variables. Chacune des variables ainsi 
  regroupées porte un nom permettant d'accéder à son espace mémoire, en vue d'y consulter la valeur 
  stockée ou de la modifier.
\end{frame}

