\begin{frame}[containsverbatim]
  \frametitle{\secname}
  \framesubtitle{\subsecname~IV}
  
  \par
  Le type \verb|void| n'a de sens que sous la forme de pointeur ou pour dire qu'une fonction ne retourne pas de valeur.
  \vspace{0.5cm}
  \par
  Le type booléen n'existe pas en C89. On utilise généralement un \verb|char|. En C99, on peut avoir un type \verb|bool| en incluant la bibliothèque \verb|<stdbool.h>| (non disponible en C++).
  \vspace{0.5cm}
  \par
  La taille d'un type, exprimée en octets, peut être connue via l'opérateur \texttt{sizeof}.
  \vspace{0.5cm}
  \par
  Les bibliothèques \verb|<limits.h>| et \verb|<float.h>| contiennent les valeurs limites de chaque type primitif.
\end{frame}

