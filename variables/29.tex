\begin{frame}[containsverbatim]
  \frametitle{\secname}
  \framesubtitle{\subsecname~I}

  La bibliothèque \verb|<stdint.h>| définit de nouveaux types qui permettent de s'abstraire de la taille 
  réelle en mémoire des types primitifs (qui peut varier suivant les plateformes de compilation).
  \begin{center}
    {\footnotesize\begin{tabular}{|c|c|c|c|}
      \hline
      Type & Taille & Min & Max \\
      \hline
      int8\_t & 8 bits & -128 & 127 \\
      \hline
      uint8\_t & 8 bits & 0 & 255 \\
      \hline
      int16\_t & 16 bits & -32768 & 32767 \\
      \hline
      uint16\_t & 16 bits & 0 & 65535 \\
      \hline
      int32\_t & 32 bits & -2147483648 & 2147483647 \\
      \hline
      uint32\_t & 32 bits & 0 & 4294967295 \\
          \hline
      int64\_t & 64 bits & -9223372036854775808 & 9223372036854775807 \\
      \hline
      uint64\_t & 64 bits & 0 & 1844674407370955161 \\
      \hline
    \end{tabular}}
  \end{center}
  \vspace{0.3cm}
  \par
  L'usage de ces types complexifie cependant légèrement l'usage des fonctions des familles \verb|printf()| et \verb|scanf()| que
  nous verrons plus tard.  
\end{frame}

