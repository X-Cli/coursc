\begin{frame}[containsverbatim]
  \frametitle{\secname}
  \framesubtitle{\subsecname~: Les constantes littérales~I} 

  Outre les constantes nommées, il est possible d'utiliser des constantes littérales.
  \vspace{0.3cm}
  \par
  Ces constantes sont également appelées ``valeurs en dur'' (hardcoded).
  \vspace{0.3cm}
  \par
  Il s'agit de constantes dont la valeur est directement exprimée dans le code.
  \vspace{0.3cm}
  \par
  Par défaut, tout nombre sans décimale ni exposant exprimé dans le code est une constante littérale de type \texttt{int}.
  \begin{exampleblock}{0 est une constante littérale de type int}
    \begin{verbatim}
      return 0;\end{verbatim}
  \end{exampleblock}
\end{frame}

