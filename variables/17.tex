\begin{frame}
  \frametitle{\secname}
  \framesubtitle{\subsecname~III}
  
  {\footnotesize\begin{tabular}{|>{\centering\arraybackslash}m{1.5cm}|>{\centering\arraybackslash}m{1cm}|>{\centering\arraybackslash}m{3cm}|>{\centering\arraybackslash}m{3cm}|}
\hline
long long (int) & 64 bits \linebreak ou \linebreak plus &  \multicolumn{2}{|c|}{ça dépend} \\
\hline
unsigned long long (int) & 64 bits \linebreak ou \linebreak plus &  \multicolumn{2}{|c|}{ça dépend} \\
\hline
float & 32 bits &  1.18e-38 &  3.4e38 \\
\hline
double & 64 bits & 2.2250738585072014 \linebreak e-308 & 1.7976931348623157 \linebreak e308 \\
\hline
long double & \multicolumn{3}{|c|}{ça dépend} \\
\hline
  \end{tabular}}
  \vspace{0.5cm}
  \par
  Tous les types, primitifs ou composés peuvent être dérivés en pointeurs, de taille 32 ou 64 bits suivant l'architecture de la machine.
\end{frame}

