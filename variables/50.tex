\begin{frame}[containsverbatim]
  \frametitle{\secname}
  \framesubtitle{\subsecname~: Les énumérations~II} 
  
  Le défaut de cette approche est que le typage est plutôt faible~: la couleur est un entier, pas une couleur. Si
  cela ne change rien au niveau technique, c'est le niveau sémantique (et donc la lisibilité) qui en pâtit.
  \vspace{0.3cm}
  \par
  Prenons l'exemple d'une fonction modifiant la couleur d'une voiture fournie en paramètre. Son prototype serait~:
  \begin{verbatim}
int setCarColor(struct Car *car, int c);\end{verbatim}
  \vspace{0.3cm}
  \par
  On peut s'interroger sur cet \verb|int| utilisé comme paramètre.
  \par
  Par ailleurs, il faut définir les couleurs unes à unes, en prenant en charge leur numérotation manuellement.
  Tout ceci est fastidieux et peu expressif.
\end{frame}

