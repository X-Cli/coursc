\begin{frame}
  \frametitle{\secname}
  \framesubtitle{\subsecname~: La déclaration~II}

  La forme la plus simple de déclaration est de mettre sur une ligne le type de la variable, suivi de l'identificateur de cette variable, puis un point-virgule.
  \vspace{0.5cm}
  \par
  Il est possible de définir plusieurs variables d'un même type lors d'une même déclaration en séparant les identificateurs par des virgules.
  \vspace{0.5cm}
  \par
  L'identificateur peut être suivi d'une paire de crochets pour définir un tableau, ces crochets pouvant contenir une constante (ou une expression, en C99) définissant la taille du tableau.
\end{frame}

