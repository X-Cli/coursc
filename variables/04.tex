\begin{frame}
  \frametitle{\secname}
  \framesubtitle{\subsecname~III}

  Les variables sont parfois nommées selon la \href{http://fr.wikipedia.org/wiki/Notation_hongroise}{\textcolor{blue}{\underline{notation hongroise}}}~: 
  on préfixe le nom de la variable par une chaine d'un ou deux caractères rappelant le type de la variable.
  \vspace{0.4cm}
  \par
  Les constantes ne suivent pas la notation CamelCase ; leur nom est uniquement composé de majuscules et les mots sont séparés par des underscores~: \texttt{MAX\_LENGTH}
%   \vspace{0.4cm}
%   \par
%   Les fonctions portent en général un nom exprimant une action. Le nom commence donc souvent par un verbe à l'infinitif~: \texttt{showEditForm()}
%   \vspace{0.4cm}
%   \par
%   Les fonctions sans effet de bord (calcul, \ldots) peuvent porter un nom qui n'est pas une action~: \texttt{sqrt()}
  \vspace{0.4cm}
  \par
  Un bon identificateur est un nom \textbf{court mais explicite}.
\end{frame}

