\begin{frame}[containsverbatim]
  \frametitle{\secname}
  \framesubtitle{\subsecname~: Les constantes littérales~V}

  Les constantes littérales de type entier peuvent être exprimées en base 10, mais également dans d'autres bases.
  \vspace{0.3cm}
  \par
  Il est possible d'exprimer des constantes en octal en faisant débuter la constante par un 0.
  \begin{exampleblock}{Constante littérale en octal}
    \begin{verbatim}
int a = 010; // a vaut 8 en décimal !\end{verbatim}
  \end{exampleblock}
  \par
  Les constantes peuvent également être exprimées en héxadécimal, en préfixant le nombre avec \verb|0x|.
  \begin{exampleblock}{Constante littérale en octal}
    \begin{verbatim}
int a = 0x1a; // a vaut 26 en décimal !\end{verbatim}
  \end{exampleblock}
\end{frame}

