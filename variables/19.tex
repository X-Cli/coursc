\begin{frame}
  \frametitle{\secname}
  \framesubtitle{\subsecname~V}
  
  \par
  Les valeurs limites peuvent être également retrouvées par simple calcul ! Inutile de mémoriser des séquences de plus de 10 chiffres !
  \vspace{0.2cm}
  \par
  Une valeur sur 8 bits est stockée sur huit 0 ou 1. Cela signifie que la valeur maximale est \textbf{11111111}. Chaque chiffre représente une puissance de 2, en commençant par $2^{0}$. Les indices vont donc de $2^{0}$ à $2^{7}$ pour une valeur sur 8 bits.
  \vspace{0.2cm}
  \par
  La valeur $2^{8}$ vaut \textbf{100000000}. La valeur maximale stockée sur 8 bits est donc $2^{8} - 1$, soit 255.
  \begin{block}{Valeur minimale et maximale}
    La valeur maximale en mode non signé est $2^{nb de bits} - 1$ et la valeur minimale est 0.
    \vspace{0.2cm}
    \par
    La valeur maximale en mode signé est $(2^{nb de bits} - 2)/2$ et la valeur minimale $-1 * (2^{nb de bits} / 2)$.
  \end{block}
\end{frame}

