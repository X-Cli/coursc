\begin{frame}[containsverbatim]
  \frametitle{\secname}
  \framesubtitle{\subsecname}

  On peut définir de ``nouveaux'' types, avec pour avantages~:
  \begin{itemize}
    \item le renforcement du typage de votre programme
    \item l'auto-documentation par le nom du type
    \item une simplicité de refactorisation (i.e. changement d'un type pour passer à un autre (e.g. valeurs limites plus grandes))
  \end{itemize}
  \begin{exampleblock}{Exemple de définition de types}
    \begin{verbatim}
// Syntaxe : typedef nouveauType typeRedéfini;
typedef unsigned int mazewidth_t;
typedef unsigned int mazeheight_t;
void build(mazewidth_t w, mazeheight_t h);\end{verbatim}
  \end{exampleblock}
  \par
  \texttt{typedef} devient encore plus pratique lorsqu'on ``aliase'' des pointeurs ou des types composés.
\end{frame}

