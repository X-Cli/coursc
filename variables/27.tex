\begin{frame}[containsverbatim]
  \frametitle{\secname}
  \framesubtitle{\subsecname~: Les constantes littérales~IV}

  Comme les caractères sont stockés sous une forme numérique, il est possible d'effectuer des opérations 
  mathématiques dessus.
  \begin{exampleblock}{Exemple d'addition sur un caractère}
    \begin{verbatim}
char monCaractere = 'C' + 2; // monCaractere vaut 'E'
    \end{verbatim}
  \end{exampleblock}
  \begin{alertblock}{Valable avec certains jeu de caractères uniquement~!}
    Attention~! Ce type d'opération n'est valable qu'avec certains jeux de caractères, comme l'ASCII.
        Cette exemple n'a comme vocation que d'illustrer que les caractères sont stockés sous forme numérique.
  \end{alertblock}
\end{frame}

