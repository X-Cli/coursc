\begin{frame}[containsverbatim]
  \frametitle{\secname}
  \framesubtitle{\subsecname~: Les constantes nommées~I} 

  Plus tôt, nous avons vu qu'il était possible de déclarer des variables avec le mot-clé \verb|const| devant le type.
  \vspace{0.5cm}
  \par
  Il est nécessaire d'initialiser la constante à la déclaration, sans quoi la valeur constante sera une valeur indéterminée~!
  \vspace{0.5cm}
  \par
  Il est également possible qu'un paramètre de fonction soit une constante. Dans ce cas, la valeur de l'argument est recopiée dans le paramètre, 
  et sa valeur ne pourra être changée pendant l'exécution de la fonction.
  \begin{alertblock}{Passage de paramètres par valeur}
     Attention~! Cela n'a rien à voir avec le fait que les arguments d'une fonction soient passés par valeur~!
  \end{alertblock}

\end{frame}

