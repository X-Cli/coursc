\begin{frame}[containsverbatim]
  \frametitle{\secname}
  \framesubtitle{\subsecname~: Les unions~I}
  Pour stocker en mémoire une structure, la place occupée par chacun de ses membres est sommée afin de déterminer
  la taille totale que chaque variable de ce type structuré occupe. S'y ajoute, pour une sombre histoire 
  d'\href{http://fr.wikipedia.org/wiki/Alignement_en_m\%C3\%A9moire}{\textcolor{blue}{\underline{alignement mémoire}}},
  quelques octets dans certains cas, mais nous ignorerons ce détail, pour des raisons de simplification. 
  \vspace{0.5cm}
  \par
  Ainsi, pour une structure X, composée de deux membres a (un \verb|int|) et b (un \verb|double|), la taille 
  d'une variable Y de type \verb|struct X| est de 12 octets (taille d'un int + taille d'un double, soit $4 + 8$).
  \vspace{0.5cm}
  \par
  Les membres \verb|Y.a| et \verb|Y.b| occupent deux emplacements de 4 et 8 octets en mémoire distincts et contigües 
  et pouvant stocker deux valeurs indépendantes.
\end{frame}   

