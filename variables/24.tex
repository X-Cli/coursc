\begin{frame}[containsverbatim]
  \frametitle{\secname}
  \framesubtitle{\subsecname~: Les constantes littérales~II}
  
  Les nombres exprimés avec des décimales ou avec un exposant sont des constantes littérales de type \texttt{double}.
  \par
  \vspace{0.5cm}
  Le chiffre suivant le \texttt{e} est la puissance de 10 par laquelle est multipliée le nombre.

  \begin{exampleblock}{Exemple de constantes littérales de type double}
    1.5
    \par
    12.98765
    \par
    1e3 // vaut $1 * 10^{3}$, soit 1000
    \par
    1.2345e2 // vaut $1,2345 * 10^{2}$, soit 123,45
    \par
    12345e-3 // vaut $12345 * 10^{-3}$, soit 12.345
  \end{exampleblock}

\end{frame}

