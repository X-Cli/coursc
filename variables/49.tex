\begin{frame}[containsverbatim]
  \frametitle{\secname}
  \framesubtitle{\subsecname~: Les énumérations~I} 

  Il est souvent souhaitable d'abstraire un mécanisme sous-jacent pour rendre le code plus expressif.
  \vspace{0.3cm}
  \par
  Par exemple, imaginons que l'on parle d'une structure représentant une voiture. L'une des caractéristiques
  de la voiture (et donc un des membres de sa structure) est sa couleur de carrosserie. Nous allons probablement
  utiliser un code couleur pour cela (0 = rouge, 1 = bleu, 2 = vert, \ldots).
  \vspace{0.3cm}
  \par
  Afin de rendre le code plus lisible, et ne pas retrouver des 0, 1, 2, \ldots dans notre code, le premier réflexe
  pourrait être de créer des constantes globales, ou des macros.
  \begin{exampleblock}{Constantes colorées}
    \begin{verbatim}    
#define COULEUR_VOITURE_BLEU 1 
// ou 
const int COULEUR_VOITURE_BLEU = 1;\end{verbatim}
  \end{exampleblock}
\end{frame}

