\begin{frame}
  \frametitle{\secname}
  \framesubtitle{\subsecname}
  Devant la difficulté de coder de grosses applications en assembleur, \textbf{les langages de haut niveau} ont été créés, selon différents paradigmes.
  \begin{itemize}
    \item Procéduraux : C, Pascal, PHP4\ldots
    \item Fonctionnels : Ocaml, Lisp, Erlang\ldots
    \item Orientés objets : C++, Java, PHP5+, Python, Ocaml\ldots
    \item Déclaratifs : SGML, XML (xHTML, DocBook, SOAP\ldots)\ldots
    \item Orientés prototypes : Javascript, Actionscript\ldots
  \end{itemize}
  \par
  Le C est considéré par certains comme un langage de bas niveau. 
  \par
  Les méchanismes de sécurité et d'abstraction y étant minimaux (par défaut), le C est principalement utilisé lorsque \textbf{la performance est un enjeu majeur} ou 
  lorsqu'un \textbf{contrôle fin de la machine} est nécessaire.
\end{frame}

