\begin{frame}
  \frametitle{\secname}
  \framesubtitle{\subsecname~: L'interprétation}
  \par
  Un interpréteur lit le code source d'un programme en langage interprété et l'analyse.
  \par
  Les instructions du langage interprété déclenchent alors l'exécution de portions de code de l'interpréteur. 
  \par
  Le processeur exécute donc du code de l'interpréteur, sur les directives du langage interprété.
  \par
  L'étape de lecture et d'analyse du code interprété est effectué à \textbf{chaque} exécution, causant un surcoût en temps de calcul, 
  et amoindrissant fortement les performances de ces applications.
  \par
  La contre-partie est que le code interprété est ``managé'' par l'interpréteur, permettant (du moins, plus facilement) des
  fonctionnalités telles que le \emph{ramasse-miette (Garbage Collecting)} et souvent une plus grande expressivité.
\end{frame}

