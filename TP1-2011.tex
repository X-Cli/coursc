\documentclass[11pt]{article}

\usepackage[french]{babel}
\usepackage[T1]{fontenc}
\usepackage[utf8]{inputenc}

\addtolength{\hoffset}{-1in}
\addtolength{\voffset}{-1.5in}
\setlength{\oddsidemargin}{1cm}
\setlength{\textwidth}{19cm}

\setlength{\headheight}{0pt}
\setlength{\topmargin}{1cm}
\setlength{\textheight}{27cm}

\pagestyle{empty}

\title{TP 1}
\author{Florian Maury}


\begin{document}

\maketitle

\section{Démarrer avec Visual Studio 2010}
\begin{enumerate}
  \item Démarrer Visual Studio C++ 2010
  \item Choisir C++
  \item Créer un projet par programme/exercice
  \item Fichier -> Nouveau -> Projet -> Projet vide
  \item Sur ``fichiers sources'', clic-droit -> ajouter -> fichier -> fichier .cpp
\end{enumerate}

\section{Exercice}
\'Ecrire un programme permettant de calculer le reste de la division d'un premier nombre par un second. \\
Vous devez coder cette fonctionnalité en utilisation ni / ni \%. \\
Tous les cas doivent être envisagés.
\par
Pour saisir les deux nombres, vous pouvez déclarer deux \verb|int|, et utiliser \verb|scanf()| de la façon suivante~:
\begin{verbatim}
int leftOperand, rightOperand, result, reminder;
scanf("%d", &leftOperand);
scanf("%d", &rightOperand);
\end{verbatim}
\par
L'affichage de la réponse peut être fait avec le code suivant~:
\begin{verbatim}
printf("%d divisé par %d donne %d et un reste de %d\n", leftOperand, rightOperand, result, reminder);
\end{verbatim}

Les fonctions \verb|printf()| et \verb|scanf()| nécessitent l'inclusion de la bibliothèque $<$stdio.h$>$.

\section{Exercice~: Jeu de la puce}
Une puce se déplace linéairement et toujours dans le même sens. Elle part de la position zéro. Sur son déplacement, des trous sont placés aux positions multiples de 7 et aux positions multiples de 3. Le joueur doit saisir la longueur du saut qui devra être comprise en 1 et 9 inclus.
La partie s’achève lorsque la puce tombe dans un trou (partie perdue) ou lorsqu’elle a réussi à faire 10 sauts sans tomber (partie gagnée).
Ecrire l’algorithme de ce petit jeu en prenant soin de ne pas afficher la position courante de la puce, mais en inscrivant à chaque itération le nombre de saut. Veillez à ce que l’utilisateur ne puisse saisir de valeurs incohérentes.


\section{Exercice}
Le programme suivant comporte une boucle infinie. Mais lors de l’exécution celui-ci s’arrête.

\begin{verbatim}
#include <stdio.h> 

int main() {
  int i; 
  i=1; 
  do {
    i=2*i;
  } while (i<2*i);
  printf("i=%d\t2*i=%d\n",i,2*i); 
}
\end{verbatim}

\begin{enumerate}
  \item Taper et exécuter ce programme. 
  \item Pourquoi l’exécution se termine-t-elle ? 
  \item En déduire un programme pour déterminer la plus grande et la plus petite valeur que l’on peut stocker dans le type int. 
  \item Quels sont ces valeurs ? 
  \item Quel est le temps d’exécution de ce programme ? (utiliser la bibliothèque $<$windows.h$>$)
  \item Proposer un second programme basé sur celui que vous venez de réaliser pour augmenter son efficacité.
\end{enumerate}


\section{Exercice~: Les montagnes russes}
Ecrire un programme qui génère l’affichage suivant :
\begin{verbatim}
   *      *      *
  ***    ***    ***
 *****  *****  *****
*********************
\end{verbatim}
Proposer une version qui demande à l’utilisateur le caractère à afficher, la hauteur des montagnes et le nombre de sommets.

\end{document}
