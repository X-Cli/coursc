\begin{frame}[containsverbatim]
  \frametitle{\secname}
  \framesubtitle{\subsecname~: if~III}

  Si parmi les instructions du block \verb|if| se trouvent des instructions mettant fin à un flot d'instructions, alors le code ``instructions 3'' n'est
  pas exécutés.
  {\small\begin{exampleblock}{Exemple d'interruption de fonction dans un block if}
    \begin{verbatim}
int fonctionDeTest(int v) {
  puts("Toujours exécuté");
  if( v < 5 ) {
    puts("V est inférieur à 5");
    return true; // sortie de fonction immédiate !
  }
  puts("Exécuté que si v n'est pas < à 5.");
  return false; // sortie de fonction immédiate !
}\end{verbatim}
  \end{exampleblock}}
  \vspace{0.3cm}
  \par
  D'autres instructions telles que \verb|break|, \verb|continue|, ou des appels à des fonctions mettant fin au programme, tels que \verb|exit()| ou \verb|abort()|
  peuvent interrompre une fonction lors de son exécution.
\end{frame}

