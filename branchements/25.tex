\begin{frame}[containsverbatim]
  \frametitle{\secname}
  \framesubtitle{\subsecname~I}

  La boucle \verb|for| est la boucle la plus versatile en C. Elle est composée de 3 parties : \textbf{l'initialisation}, qui est exécuté une seule fois, au début de 
  la boucle, \textbf{la condition}, qui est évaluée à chaque tour de boucle (y compris à l'entrée dans la boucle, juste après l'initialisation), et une troisième partie, 
  exécutée à chaque fin de tour de boucle, juste avant que la condition ne soit réévaluée.
  \vspace{0.5cm}
  \begin{block}{Syntaxe générale d'une boucle for}
    \begin{verbatim}
for( initialisation ; condition ; incrémentation) {
  //instructions
}\end{verbatim}
  \end{block}
\end{frame}

