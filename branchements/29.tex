\begin{frame}[containsverbatim]
  \frametitle{\secname}
  \framesubtitle{\subsecname~V}

  Chaque partie du for peut être vide.
  \par
  Si la partie initialisation est vide, il n'y a pas d'initialisation avant l'exécution de la boucle.
  \par
  Si la partie incrémentation est vide, il n'y a pas d'incrémentation avant l'évaluation de la condition, à chaque tour de boucle.
  \par
  Si la partie condition est vide, la condition est évaluée à vraie. On obtient une boucle infinie.
  
  {\small\begin{exampleblock}{Boucle infinie}
    \begin{verbatim}
for(;;) {
  // instructions exécutées à l'infini
}\end{verbatim}
  \end{exampleblock}}
  \begin{alertblock}{Boucles infinies}
    Les boucles infinies sont à éviter absolument. Elles sont employées dans certains cas d'optimisation algorithmique avancée.
  \end{alertblock}
\end{frame}

