\begin{frame}[containsverbatim]
  \frametitle{\secname}
  \framesubtitle{\subsecname~VII}

  Si un \verb|case| ne se termine pas par un \verb|break|, le code n'en devient pas invalide. L'exécution continuera en exécutant le code dans le case suivant, jusqu'à
  rencontrer un \verb|break|, ou la fin du block \verb|switch|.
  \vspace{0.3cm}
  \par
  On dit que l'exécution \textit{déborde} dans le \verb|case| suivant.
  {\footnotesize\begin{exampleblock}{Débordement d'un cas}
    \begin{verbatim}
switch(a) {
  case 1:              // si a == 1, les instructions 1 et 2 
    //instructions 1   // sont exécutées
  case 2:              // si a == 2, seules les instructions 2
    //instructions 2   // sont exécutées
    break;
  default:             // si a n'est pas égal à 1, 2 ou 3, alors
    //instructions 3   // les instructions 3 et 4 sont exécutées
  case 3:              // si a == 3, seules les instructions 4 
    //instructions 4   // sont exécutées
}\end{verbatim}  
  \end{exampleblock}}
\end{frame}

