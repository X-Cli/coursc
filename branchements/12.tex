\begin{frame}[containsverbatim]
  \frametitle{\secname}
  \framesubtitle{\subsecname~: Code mort}

  \begin{alertblock}{Attention au code mort}
    Les branchements conditionnels peuvent devenir vite complexes à force d'imbrication. Il faut alors veiller à ne pas créer du \textbf{code mort}.
    \vspace{0.3cm}
    \par
    Du code mort est du code qui ne sera jamais exécuté, quoiqu'il arrive. Ce cas intervient notamment si la condition décidant si une série d'instructions est exécutée
    ne peut jamais devenir vraie. La détection de ce code n'est pas toujours triviale comme dans l'exemple ci-dessous.
  \end{alertblock}
  {\footnotesize\begin{exampleblock}{Exemple de code mort}
    \begin{verbatim}
if(a < 5) {
  if(a > 50) {
    //ce code ne sera jamais exécuté 
  }
  // instructions
}\end{verbatim}
  \end{exampleblock}}
\end{frame}

