\begin{frame}[containsverbatim]
  \frametitle{\secname}
  \framesubtitle{\subsecname~: if~I}

  Il est possible de diviser le flot d'exécution en branches distinctes suivant une condition~: une expression évaluée à vrai ou faux.
  \begin{block}{Syntaxe générale d'un block if}
    \begin{verbatim}
// instructions 1
if( expression ) {
  // instructions 2
}
// instructions 3\end{verbatim}
  \end{block}
  \par
  Une fois les instructions ``instructions 1'' exécutées, l'expression entre parenthèses est évaluée. 
  \par
  Si elle s'avère vraie, alors les instructions ``instructions 2'' sont exécutées à leur tour, puis les instructions ``instructions 3'' le sont également.
  \par 
  Si la condition s'avère fausse, alors les instructions ``instructions 2'' NE sont PAS exécutées puis les instructions ``instructions 3'' sont exécutées.
\end{frame}

