\documentclass[10pt]{article}

\usepackage[french]{babel}
\usepackage[T1]{fontenc}
\usepackage[utf8]{inputenc}
\usepackage{acronym}

\addtolength{\hoffset}{-1in}
\addtolength{\voffset}{-1.5in}
\setlength{\oddsidemargin}{1cm}
\setlength{\textwidth}{19cm}

\setlength{\headheight}{0pt}
\setlength{\topmargin}{1cm}
\setlength{\textheight}{27cm}

\pagestyle{empty}

\title{TP 2}
\author{Florian Maury}

\begin{document}
\section{LFSR~: écriture d'un générateur de nombres pseudo-aléatoires}
\begin{verbatim}
bool getRandomBit(unsigned short seed);
\end{verbatim}


La fonction getRandomBit() peut être appelée de deux manières différentes~:
\begin{itemize}
  \item Son paramètre est non nul~: dans ce cas, le ``registre'' du générateur est initialisé avec la valeur fournie en paramètre
  \item Son paramètre est nul~: le générateur utilise la valeur contenue actuellement dans le ``registre''
\end{itemize}

Le registre est une variable statique locale à la fonction getRandomBit() car la valeur doit être préservée d'un appel sur l'autre de la fonction.

Cette fonction utilise l'implémentation de Fibonacci de LFSR.
A chaque appel, la fonction initialise le registre avec la valeur fournie en paramètre si elle est non nulle.
Elle sauve ensuite la valeur du bit de poids le plus faible du registre. C'est cette valeur qui sera renvoyée par la fonction à la fin du traitement.
Les bits aux rangs 0, 2, 3, et 5 (le bit de poids le plus fort est de rang 15 et celui de poids le plus faible 0) sont XORés ensembles.
Le registre est alors décalé (binairement) une fois à droite. Le bit généré par les XORs est OR avec le bit de poids le plus fort.

\subsection{Correction}
\begin{verbatim}
bool getRandomBit(unsigned short seed) {
  static unsigned short s = 0;
  bool bGeneratedBit;

  // Seeding the generator
  if(s == 0) {
    if(seed == 0) {
      s = (unsigned short) time(NULL);
    }
    else {
      s = seed;
    }
  } else if (seed != 0) {
    s = seed;
  }
  
  //bGeneratedBit = s >> 15 & 1;
  //s = (s << 1) | ( ( ((s>>15) & 1) ^ ((s>>13) & 1) ^ ((s>>12) & 1) ^ ((s>>10) & 1)) & 1 );

  bGeneratedBit = s & 1;
  s = (s >> 1) | ( ((s & 1) ^ ((s >> 2) & 1) ^ ((s >> 3) & 1) ^ ((s >> 5) & 1)) << 15);
  return bGeneratedBit;
}
\end{verbatim}

\section{Générer un int pseudo-aléatoire}
\'Ecrire un programme qui génère un entier 32 bits non signé pseudo-aléatoire à l'aide de la fonction getRandomBit précédemment écrite, initialisée avec le seed ``0x4242'' et l'affiche à l'écran à l'aide de la ligne suivante~:
\begin{verbatim}
printf("%d\n", entierAleatoire);
\end{verbatim}

\subsection{Correction}
\begin{verbatim}
int main() {
  unsigned int iRandomValue = (unsigned int) getRandomBit(INIT_VALUE);
  for(int i = 1 ; i < 32 ; i++) {
    iRandomValue = (iRandomValue<<1) | (unsigned int) getRandomBit(0);
  } 
  printf("%d\n", iRandomValue);
  return 0; 
}
\end{verbatim}

\section{Fonction getRandomInt} 
\'Ecrire une fonction qui renvoie un entier 32 bits pseudo-aléatoire. Cette fonction doit être une généralisation de la fonction du premier exercice. 
La solution doit être plus performante que générer 32 bits en faisant 32 appels à getRandomBit (en gros, faut utiliser une boucle\ldots).

\subsection{Correction}
\begin{verbatim}
unsigned int getRandomInt(unsigned short seed) {
  static unsigned short s = 0;
  unsigned int iGeneratedInt = 0;

  // Seeding the generator
  if(s == 0) {
    if(seed == 0) {
      s = (unsigned short) (time(NULL));
      printf("init with time: %d\n", s);
    }
    else {
      s = seed;
    }
  } else if (seed != 0) {
    s = seed;
  }
  
  for(int i = 0 ; i < 32 ; i++) { 
    iGeneratedInt = (iGeneratedInt << 1) | (s & 1);
     s = (s >> 1) | ( ((s & 1) ^ ((s >> 2) & 1) ^ ((s >> 3) & 1) ^ ((s >> 5) & 1)) << 15);
    //iGeneratedInt = (iGeneratedInt << 1) | ((s >> 15) & 1);
    //s = (s << 1) | ( ( ((s>>15) & 1) ^ ((s>>13) & 1) ^ ((s>>12) & 1) ^ ((s>>10) & 1)) & 1 );
  }
  return iGeneratedInt;
}
\end{verbatim}

\section{Jouer avec la génération pseudo-aléatoire}
\'Ecrire un programme qui génère un unsigned char aléatoire. Le programme vous propose ensuite de saisir au clavier une valeur entre 0 et 255 et vous indique si votre saisie est plus grande ou moins grande que le nombre pseudo-aléatoire. Si votre saisie est égale au nombre pseudo aléatoire, le programme vous affiche le message "Victoire" et quitte.
Pour rendre le jeu intéressant, initialiser le générateur pseudo-aléatoire avec le résultat de la fonction time() (casté en unsigned short) de la bibliothèque $<$time.h$>$
\par
La saisie au clavier se fait à l'aide de la ligne suivante. La variable ``input'' doit être de type short !
\begin{verbatim}
fflush(stdin);
scanf("%hd", &input);
\end{verbatim}

\subsection{Correction}
\begin{verbatim}
int main() {
  unsigned char ucToFind = getRandomUChar(0);
  short sGuess;

  printf("Cheatcode: %d\n", ucToFind);
  do { 
    do {
      puts("Try to guess:");
      fflush(stdin);
      scanf("%hd", &sGuess);
    } while(sGuess < 0 || sGuess > 255);
    if(sGuess < ucToFind) {
      puts("Too low");
    }
    else if(sGuess > ucToFind) {
      puts("Too high");
    }
  } while(sGuess != ucToFind);
  puts("Victoire !!");
  return 0; 
}
\end{verbatim}

%\section{Décoder le message secret}
%\'Ecrire un programme qui demande des entiers 32 bits pseudo-aléatoires (initialisation avec 0x4242) et les XOR avec des valeurs saisies au clavier. Ces valeurs sont des int, saisis avec la ligne de commande suivante :
%\begin{verbatim}
%fflush(stdin);
%scanf("%d", &input);
%\end{verbatim}
%
%Afficher le résultat du XOR avec la ligne de commande suivante~:
%\begin{verbatim}
%printf("%c%c%c%c\n", (input>>24), ((input >> 16) & ((1<<8) - 1)), ((input >> 8) & ((1<<8) - 1)), (input & ((1<<8) - 1)));
%\end{verbatim}
%
%Voici la liste des entiers à saisie au clavier~:
%\begin{itemize}
%  \item 123
%\end{itemize}  

%#ifndef __cplusplus
%#include <stdbool.h>
%#endif
%
%#include <stdio.h>
%#include <time.h>
%#include <stdlib.h>
%#define INIT_VALUE 0x4242
%
%bool getRandomBit(unsigned short seed) {
%  static unsigned short s = 0;
%  bool bGeneratedBit;
%
%  // Seeding the generator
%  if(s == 0) {
%    if(seed == 0) {
%      s = (unsigned short) time(NULL);
%    }
%    else {
%      s = seed;
%    }
%  } else if (seed != 0) {
%    s = seed;
%  }
%
%  bGeneratedBit = s >> 15 & 1;
%  s = (s << 1) | ( ( ((s>>15) & 1) ^ ((s>>13) & 1) ^ ((s>>12) & 1) ^ ((s>>10) & 1)) & 1 );
%  return bGeneratedBit;
%}
%
%int getRandomInt(unsigned short seed) {
%  static unsigned short s = 0;
%  int iGeneratedInt = 0;
%
%  // Seeding the generator
%  if(s == 0) {
%    if(seed == 0) {
%      printf("init with time");
%      s = (unsigned short) time(NULL);
%    }
%    else {
%      s = seed;
%    }
%  } else if (seed != 0) {
%    s = seed;
%  }
%
%  for(int i = 0 ; i < 32 ; i++) {
%    //iGeneratedInt = (iGeneratedInt << 1) | (s & 1);
%    iGeneratedInt = (iGeneratedInt << 1) | ((s >> 15) & 1);
%    s = (s << 1) | ( ( ((s>>15) & 1) ^ ((s>>13) & 1) ^ ((s>>12) & 1) ^ ((s>>10) & 1)) & 1 );
%  //  s = (s >> 1) | ( ( ((s>>15) & 1) ^ ((s>>13) & 1) ^ ((s>>12) & 1) ^ ((s>>10) & 1) ^ (s & 1) ) << 15 );
%  }
%  return iGeneratedInt;
%}
%
%unsigned char getRandomUChar(unsigned short seed) {
%  static unsigned short s = 0;
%  unsigned char ucGeneratedUChar = 0;
%
%  // Seeding the generator
%  if(s == 0) {
%    if(seed == 0) {
%      s = (unsigned short) (time(NULL));
%      printf("init with time: %d\n", s);
%    }
%    else {
%      s = seed;
%    }
%  } else if (seed != 0) {
%    s = seed;
%  }
%
%  for(int i = 0 ; i < 32 ; i++) {
%    ucGeneratedUChar = (ucGeneratedUChar << 1) | ((s >> 15) & 1);
%    s = (s << 1) | ( ( ((s>>15) & 1) ^ ((s>>13) & 1) ^ ((s>>12) & 1) ^ ((s>>10) & 1)) & 1 );
%  }
%  return ucGeneratedUChar;
%}
%
%int main() {
%  unsigned char ucToFind = getRandomUChar(0);
%  short sGuess;
%  unsigned int iRandomValue = (unsigned int) getRandomBit(INIT_VALUE);
%  for(int i = 1 ; i < 32 ; i++) {
%    iRandomValue = (iRandomValue<<1) | (unsigned int) getRandomBit(0);
%  }
%  printf("%d\n", iRandomValue);
%  printf("%d\n", getRandomInt(INIT_VALUE));
%  printf("%d\n", getRandomInt(0x1010));
%  printf("%d\n", getRandomInt(0));
%
%
%  printf("Cheatcode: %d\n", ucToFind);
%  do {
%    do {
%      puts("Try to guess:");
%      fflush(stdin);
%      scanf("%hd", &sGuess);
%    } while(sGuess < 0 || sGuess > 255);
%    if(sGuess < ucToFind) {
%      puts("Too low");
%    }
%    else if(sGuess > ucToFind) {
%      puts("Too high");
%    }
%  } while(sGuess != ucToFind);
%  puts("Victoire !!");
%  return 0;
%}



\end{document}
