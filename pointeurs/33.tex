\begin{frame}[containsverbatim]
  \frametitle{\secname}
  \framesubtitle{\subsecname~: L'opérateur sizeof~I}

  L'opérateur sizeof permet de retourner la taille d'un type de données. La taille renvoyée est exprimée en octets.
  \vspace{0.3cm}
  \begin{exampleblock}{Utilisation de l'opérateur sizeof}
    \begin{verbatim}
sizeof int
// ou
sizeof(int)\end{verbatim}
  \end{exampleblock}
  \par
  Sur un type complexe (structures, tableaux), l'opérateur sizeof prend tout son sens car la taille n'est pas forcément facilement calculable (notamment pour les structures à cause de l'alignement mémoire).
  \vspace{0.3cm}
  \par
  Sur un tableau de taille statique déclaré localement à la fonction en cours, ou globalement, l'opérateur sizeof renvoie la 
  dimension du tableau multipliée par la taille des éléments de ce tableau.
\end{frame}

