\begin{frame}[containsverbatim]
  \frametitle{\secname}
  \framesubtitle{\subsecname~: L'opérateur ->~II}

  Afin d'alléger l'écriture, il est possible d'utiliser l'opérateur \texttt{->} sur un pointeur de structure ou d'union pour accéder au membre qui suit la flèche.
  \vspace{0.3cm}
  \par
  Autrement dit, l'opérateur \texttt{->} déréférence son opérande gauche, de type structure ou union, et accède au membre nommé comme opérande droit. 
  \begin{block}{Syntaxe générale de l'opérateur ->}
    \begin{verbatim}
pointeurSurStructureOuUnion->membre\end{verbatim}
  \end{block}
  \begin{exampleblock}{Exemple d'utilisation de l'opérateur ->}
    \begin{verbatim}
struct Date maDate = {11, 11, NULL};
struct * pMaDate = &maDate;
pMaDate->jour += 1; //équivalent à (*pMaDate).jour += 1\end{verbatim}
  \end{exampleblock}
\end{frame}

