\begin{frame}%[containsverbatim]
  \frametitle{\secname}
  \framesubtitle{\subsecname~: L'opérateur [ ]~IX}

  L'opérateur ``crochets'' permet également d'accéder aux cases d'un tableau, après la déclaration. Pour cela, on utilise l'identificateur
  du tableau suivi d'une paire de crochet contenant une expression entière. Le résultat de l'expression indique l'indice du tableau qui
  va être retourné.
  \vspace{0.5cm}
  \par
  Les indices d'un tableau, en C, commencent à 0. Ainsi pour un tableau de 4 cases, les indices valides sont 0, 1, 2 et 3 ! 
  \vspace{0.5cm}
  \par
  4 \textbf{N}'est alors \textbf{PAS} un indice valide ! Accéder à un indice invalide ne provoque pas toujours un crash de l'application,
  mais souvent des résultats imprévisibles. Il s'agit d'une erreur importante de programmation.
\end{frame}

