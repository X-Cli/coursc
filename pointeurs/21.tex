\begin{frame}%[containsverbatim]
  \frametitle{\secname}
  \framesubtitle{\subsecname~: Qu'est ce qu'un tableau ?}

  Un tableau est une zone mémoire composée de un ou plusieurs espaces mémoire contigües, chacun de la taille du type de données du tableau.
  \vspace{0.3cm}
  \par
  Ils peuvent être locaux à une fonction, globaux, statiques, alloués statiquement (taille fixée) ou dynamiquement.
  \vspace{0.3cm}
  \par
  En C89, la taille des tableaux doit être une \textbf{constante} numérique entière, ou ils doivent être alloués dynamiquement avec 
  une fonction de la famille \texttt{malloc} ou la fonction \texttt{alloca} (spécifique à GLIBC). 
  \vspace{0.3cm}
  \par
  Depuis C99 (et en C++), une expression peut être utilisée pour déclarer la taille d'un tableau.
\end{frame}

