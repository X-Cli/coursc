\begin{frame}[containsverbatim]
  \frametitle{\secname}
  \framesubtitle{\subsecname~: Les tableaux et les fonctions~I}

  Un tableau est une forme de pointeur. Si l'identificateur d'un tableau est utilisé sans l'opérateur crochets, alors l'adresse du premier
  élément du tableau est retournée.
  \vspace{0.3cm}
  \par
  De ce fait, on ne peut passer à une fonction qu'un élément du tableau (sa valeur), ou un pointeur (vers l'un des éléments du tableau).
  \vspace{0.3cm}
  \begin{exampleblock}{Un tableau en argument, un pointeur en paramètre}
    \begin{verbatim}
void exemple(int * tab);
//...
int tableau[4];
exemple(tableau);\end{verbatim}
  \end{exampleblock}
\end{frame}

