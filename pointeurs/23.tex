\begin{frame}[containsverbatim]
  \frametitle{\secname}
  \framesubtitle{\subsecname~: L'opérateur [ ]~II}

  Il est possible d'allouer des tableaux à plusieurs dimensions en chainant plusieurs opérateurs crochets.
  \vspace{0.3cm}
  \par
  Les deux dimensions n'ont pas besoin d'être identiques. En revanche, seules des matrices peuvent être ainsi déclarées (même nombre 
  de ``lignes'' dans les ``colonnes'').
  \begin{exampleblock}{Déclaration d'un tableau à 2 dimensions}
    \begin{verbatim}
int table[5][3];\end{verbatim}    
  \end{exampleblock}
  Ce tableau a 5 ``colonnes'' de 3 ``lignes''. 
  \vspace{0.3cm}
  \par
  La notion de ``colonnes'' et de ``lignes'' est une simple vue, le tableau étant aplati en 
  mémoire à une seule dimension.
\end{frame}

