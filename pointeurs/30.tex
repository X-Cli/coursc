\begin{frame}[containsverbatim]
  \frametitle{\secname}
  \framesubtitle{\subsecname~: L'opérateur [ ]~VIII}

  \begin{exampleblock}{Initialisation partielle d'un tableau avec indice}
    \begin{verbatim}
int tableau[5] = { [2] = 4, 3, 4};
//équivalent à 
int tableau[5] = {0, 0, 4, 3, 4};\end{verbatim}
  \end{exampleblock}
  \begin{alertblock}{Danger de l'initialisation avec ``crochets''}
    La syntaxe d'initialisation avec crochets est dangereuse (source d'erreur) si elle est mixée avec un tableau dont la taille n'est pas
    fixée par une expression. Dans ce cas, la taille du tableau est équivalente à l'indice le plus élevé + 1.
    \begin{verbatim}int tableau[] = { [4] = 1 };
//équivalent à 
int tableau[5] = {0, 0, 0, 0, 1};\end{verbatim}
  \end{alertblock}
\end{frame}

