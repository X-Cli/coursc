\begin{frame}[containsverbatim]
  \frametitle{\secname}
  \framesubtitle{\subsecname~: L'opérateur [ ]~VI}

  Lors de l'initialisation, une syntaxe permet de n'initialiser que certains éléments, les autres valant 0.
  \vspace{0.3cm}
  \par
  Pour cela, on place entre crochet le numéro de l'élément à initialiser et on lui affecte la valeur désirée.
  \vspace{0.3cm}
  \par
  En C, \textbf{les indices des tableaux commencent à 0}.
  \begin{exampleblock}{Initialisation partielle avec indice}
    \begin{verbatim}
int tableau[5] = { [2] = 1, [4] = 2 };
//équivalent à :
int tableau[5] = {0, 0, 1, 0, 2 };\end{verbatim}
  \end{exampleblock}
\end{frame}

