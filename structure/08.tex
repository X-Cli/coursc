\begin{frame}
  \frametitle{\secname}
  \framesubtitle{\subsecname~I}  
  \begin{block}{Vocabulaire}
    \par
    La \textbf{déclaration} d'une fonction est lorsqu'on fait apprendre au compilateur à quoi ressemble une fonction.
    \vspace{0.3cm}
    \par
    Les informations identifiant une fonction sont son type de retour, son nom, son nombre de paramètres et leurs types.
    \vspace{0.3cm}
    \par
    On appelle l'ensemble de ces informations le \emph{profil} ou le \emph{prototype} de la fonction.
    \vspace{0.3cm}
    \par
    L'\textbf{implémentation} d'une fonction est lorsqu'on rappelle le prototype d'une fonction, et qu'on définit les actions qu'elle doit entreprendre (par une série d'instructions).
  \end{block}
\end{frame}

