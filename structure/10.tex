\begin{frame}[containsverbatim]
  \frametitle{\secname}
  \framesubtitle{\subsecname~III}

  \begin{block}{Vocabulaire}
    \par
    Pour une fonction, lors de sa déclaration et de son implémentation \verb|int toto(int a, int b) { }|, on appelle \verb|a| et \verb|b| des \emph{paramètres}.
    \vspace{0.5cm}
    \par
    Lorsqu'on appelle cette fonction (\verb|toto(1, 2);|), les valeurs \verb|1| et \verb|2| sont affectées respectivement à \verb|a| et \verb|b|.
    \vspace{0.5cm}
    \par
    Les valeurs données lors d'un appel de fonction, et qui sont affectées aux paramètres sont nommées \emph{arguments} (de la fonction).
  \end{block}
\end{frame}

